
%% notes on web page

\subsection{Non-Regular Languages}
\label{subsec:nonreg_lang}
% --------------------------------------------------

Are there non-regular languages?
The answer is obviously ``Yes,'' but what are they?
Take binary strings as example, find
$L \subseteq \set{0,1}^*\ \text{s.t.}\ \forall\ \text{regex}\ E, L(E) \neq L$.
With symbols
$\set{ 0,1,\cdot,+(\cup),*,(,),\varnothing }$,
any regular language can be expressed,
\[
    E = (0+1)^*.
\]
Map each of the symbols to
$
0 = 000,
1 = 001,
\cdot = 010,
\cdots
\varnothing = 111
$
and use function $\varphi$ to rewrite the regex above,
\[
    \varphi(E) = 101\,000\,011\,001\,110\,100 \qquad \in \set{0,1}^*
\]
so
$
L(E) \subseteq \set{0,1}^*.
$
Then the non-regular language is
\[
    L = \set{ \varphi(E) \mid \varphi(E) \notin L(E) },
\]
which is called the diagonal language (see \autoref{def:diagonal_language}).

\begin{definition}[Diagonal Language]
    \label{def:diagonal_language}
    \[
        D = \set{
            \varphi(E)
            \mid
            E \in \text{RegEx}\(\set{0,1}\) , \varphi(E) \notin L(E)
        }
    \]
\end{definition}

\begin{proof}[
    Proof of $\varphi(''\varnothing'') = ''111'' \notin L(\varnothing) = \varnothing$
]
    Assume for controdiction $L$ is regular,  \\
    Let $E$ s.t.\ $L(E) = L$.  \\
    \[
        \varphi(E) \in L \leftrightarrow \varphi(E) \notin L(E) = L.
    \]
\end{proof}

\subsection{Pumping Lemma}
% --------------------------------------------------

Every regular language satisfies property $P$,
if a language $L$ does not satisfy $P$ then $L$ is non-regular.
In proving, we will first assume $P(L)$ and lead to controdiction.

\begin{definition}[Property $P$]
    \begin{align*}
        &\exists p \ge 1 \qquad \text{(pumping length)}     \\
        &\forall w \in L, \abs{w} \ge p                     \\
        &\exists x,y,z \in \Sigma^*, w = x \cdot y \cdot z
    \end{align*}
    and
    \begin{compactitem}
    \item
        $
        \forall i \ge 0, \quad
        x \cdot y^i \cdot z = x \overbrace{yy \cdots y}^i z \in L
        $,
    \item
        $
        \abs{y} \ge 0 ( y \neq \varepsilon )
        $,
    \item
        $
        \abs{xy} \le p
        $.
    \end{compactitem}
\end{definition}

