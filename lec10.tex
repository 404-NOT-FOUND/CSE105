
\subsection{FST and DFA}

Combine FST $M$ with DFA $M'$ to test the translated string.
An FST combined with a DFA is a DFA.
Notice that an FST does not have a language; rather than procedures such as DFAs, it instead describes
a function.

\begin{theorem}[]\ \\
    \label{thm:translation_of_DFA}
    $
    \forall \text{\upshape FST} F = \lst{Q_F,\Sigma,\Gamma,\delta_F,s_F},   \\
    \phantom{\forall} \text{\upshape DFA} D = \lst{Q_D,\Gamma,\delta_D,s_D,F_D},  \\
    \exists \text{\upshape DFA} M = \lst{Q,\Sigma,\delta,s,F}
    \text{ s.t.\ }
    $
    \[
        L(M) = f_M^{-1} \( L(M_D) \) \text{ meaning }
        L(M) = \set{ w \in \Sigma^* \mid f_M(w) \in L(M_D) }.
    \]
\end{theorem}

\begin{proof}[Proof of \autoref{thm:translation_of_DFA}]
    \begin{align*}
        \text{Let }
        & Q = Q_F \times Q_D,  \\
        & S = (s_F,s_D),       \\
        & F = Q_F \times F_D,  \\
        & \delta \colon Q \times \Sigma \mapsto Q \text{ s.t.\ } \\
        & \delta((q_F,q_D),a) = \big( q_F', \delta_D^*(q_D,w) \big)
        \text{ where } (q_F',w) = \delta_F(q_F,a).
    \end{align*}
\end{proof}

\begin{theorem}[]
    %% pic fst -> fst
    Compose two FST
    %% see class notes
\end{theorem}

\begin{definition}[Reduction of Languages]
    \[
        f \colon \Sigma^* \mapsto \Gamma^* \text{ s.t.\ } A = f^{-1}(B)
    \]
    is called a reduction from $A$ to $B$. If such reduction exists, we say $A$ is
    reducable to $B$, denoted by
    \[
        \reducable AB.
    \]
\end{definition}

\begin{theorem}[]\ \\
    If $\reducable AB$ and $B$ is regular, then $A$ is regular. \\
    If $\reducable AB$ and $A$ is not regular, then $B$ is not regular.
\end{theorem}

