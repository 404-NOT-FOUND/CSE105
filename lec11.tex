
\begin{example}[Application of reductions]

    $A$ and $B$ are languages where
    \begin{align*}
        A = \set{ 0^n 1^n \mid n \ge 0 } \quad \text{and} \quad
        B = \set{ 0^n 1 0^n \mid n \ge 0 }
    ,\end{align*}
    Given that $A$ is nonregular, is $B$ regular?

    \begin{proof}
        Build an FST that translates $A$ to $B$,
        \centgraph[4cm]{mp/fst-0}
        so 
        $
        A \reducable B
        .$
        Since $A$ is nonregular,
        $B$ is nonregular, too.
    \end{proof}

\end{example}

\begin{example}
    Which ones of the following statements are true?
    \begin{compactenum}
    \item If $B$ is regular then $f_M^{-1}(B)$ is regular, \label{itm:B->f-1B} 
    \item If $A$ is regular then $f_M(A)$ is regular, \label{itm:A->fA}
    \item If $A$ is nonregular then $f_M(A)$ is nonregular. \label{itm:notA->notfA}
    \end{compactenum}

    Statements \ref{itm:B->f-1B} and \ref{itm:A->fA} are true.
    Statement \ref{itm:notA->notfA} can be easily proved wrong with FST
    \centgraph[3cm]{mp/fst-1}
    which translates any string (regular or nonregular) consisted with 0s and 1s to empty
    string $\varepsilon$, thus to a regular language $\set{\varepsilon}$.
\end{example}

\section{NFST}
% ==================================================

\begin{definition}[NFST]
    A Nondeterministic Finite State Transducer (NFST) is 
    \[
        M = \lst{Q,\Sigma,\Gamma,\delta,s,F}
    \]
    where
    \begin{compactenum}
    \item $Q$      is a finite set of states,
    \item $\Sigma$ is a finite set of input  symbols,
    \item $\Gamma$ is a finite set of output symbols,
    \item 
        $
        \delta \colon
        Q \times \Sigma_\varepsilon \mapsto
        \powerset(Q \times \Gamma_\varepsilon)
        $
        is the transition function,
    \item $s \in Q$ is the start state, and
    \item $F \subseteq Q$ is a set of final states.
    \end{compactenum}
\end{definition}

An NFST outputs a set of strings on an input.

\begin{definition}[Possible Outputs]
    The set of possible outputs of $M$ on input $w$
    \[
        f_M \colon \Sigma^* \mapsto \gamma (\Gamma^*)
    \]
    is defined as
    \[
        f_M(w) = \set{ u \in \Gamma^* \mid (q,u) \in \delta^*(s,w), q \in F }
    \]
\end{definition}

\begin{theorem}\
    \begin{compactitem}
    \item If $A \reducableN B$ and $B$ is regular, then $A$ is regular,
    \item If $A \reducableN B \reducableN C$ then $A \reducableN C$,
    \item If $f = f_M$ then $\exists M' \text{ s.t.\ } f^{-1} = f_{M'}$.
    \end{compactitem}
\end{theorem}

