
% no Haskell on exams.

\section{Regular Expression}
% ==================================================

\subsection{Regular Language}
% --------------------------------------------------

We've learned that the class of regular languages is closed under
\begin{compactitem}
\item union,
    \footnote{
        $
        \text{ if } L_1, L_2 \in \mathbb R,
        \text{ then }
        L_1 \cup L_2 = \set{ w \mid w \in L_1 \lor w \in L_2 }
        \in \mathbb R
        .$
    }
\item intersection,
    \footnote{
        $
        \text{ if } L_1, L_2 \in \mathbb R,
        \text{ then }
        L_1 \cap L_2 = \set{ w \mid w \in L_1 \land w \in L_2 }
        \in \mathbb R
        .$
    }
\item concatenation, and
    \footnote{
        $
        \text{ if } L_1, L_2 \in \mathbb R,
        \text{ then }
        L_1 \cdot L_2 = \set{ w_1w_2 \mid w_1 \in L_1 \land w_2 \in L_2 }
        \in \mathbb R
        .$
    }
\item star.
    \footnote{
        $
        \text{ if } L \in \mathbb R,
        \text{ then }
        L^* = \set{ w_1w_2 \cdots w_n \mid w_1,w_2,\cdots,w_n \in L, n \ge 0 }
        \in \mathbb R
        .$
    }
\end{compactitem}

Many complex languages can be built using these operations; \autoref{exa:language_numbers}
is one practical example.

\begin{example}[Build a rather complex language]
    % any number with commas
    It is common we want to search for all numbers, say, in a file. The following set is a
    language that matches all numbers greater than $10$ and allowing appearance of commas.
    \[
        L =
        \set{1,\cdots,9} \cdot
        \( \set{0,1,\cdots,9}^* \cdot \set{,} \)^* \cdot
        \set{0,1,\cdots,9}
    \]

    Consider 
    \begin{align*}
        L_1 = \set{1,\cdots,9} \\
        L_2 = \set{0,1,\cdots,9}^* \cdot \set{,}  \\
        L_3 = \set{0,1,\cdots,9}  \\
    \end{align*}
    so 
    $L = L_1 \cdot L_2^* \cdot L_3$.

    What are $L_1$, $L_2$ and $L_3$?
    \begin{compactdesc}
    \item[$L_1$] is a set of all digits from $1$ to $9$;
    \item[$L_3$] is a set of all digits from $0$ to $9$;
    \item[$L_2$] is a little more complicated, it can also be written as $L_3^* \cdot
        \set{,}$, while $L_3^*$ matches a string of any number of elements in $L_3$, that
        is, a string made of all digits with unknown length. What $\cdot \set{,}$ does is
        it appends a comma to the end of this string. In all, $L_2$ is a number of digits
        with a comma at the end.
    \end{compactdesc}

    With that, the set $L_1 \cdot L_2^* \cdot L_3$ can be now (roughly) seen as:
    \[
        a\ digit \text{ and }
        a\ number\ of ( a\ number\ of\ digits \text{ and } a\ comma ) \text{ and }
        a\ digit
    \]
    Now, notice there is a leading digit and an ending one, why should one be in $L_1$ and
    the other $L_3$? Because matching from set $L_1$ rules out the numbers with leading
    $0$s ($L_1$ doesn't have $0$), and the rest of digits should allow $0$s. The middle
    portion ($L_2^*$) allows unknown number of strings from $L_2$ (even $0$) in between
    the first and last digit. In the case where the number of $L_2$ is $0$, which makes
    the input string also in set $L_1 \cdot L_3$, the input string is a two-digit number
    ($10$ to $99$).
\end{example}

\begin{example}[$\mathbb R$ closed under star]
    \begin{proof}
        use $\varepsilon$ transition from the final states to the initial states
        (including states transited directly from the initial state with an $\varepsilon$
        arrow) to prove that $\mathbb R$ is closed under star.
    \end{proof}
\end{example}

\subsection{Regular Expression}
% --------------------------------------------------

A regular expression (abbreviated regex or regexp) is a sequence of characters that forms
a search pattern, mainly for use in pattern matching with strings, or string matching,
i.e. "find and replace"-like operations.
(via \href{http://en.wikipedia.org/wiki/Regular_expression}{WikiPedia})

A regex $E$ has the following rules
\begin{align*}
    & E = a                   && L(a) = \set{a}                        \\
    & E = \varepsilon         && L(\varepsilon) = \set{\varepsilon}    \\
    & E = E_1 \cdot E_2       && L(E) = L(E_1) \cdot L(E_2)            \\
    & E = E_1 + E_2           && L(E) = L(E_1) + L(E_2)                \\
    & E = (E_1)^*             && L(E) = L\((E_1)^*\) = L(E_1)^*        \\
    & E = \varnothing         && L(\varnothing^*)
\end{align*}
In fact, the rule
\[
    L(\varepsilon) = \set{\varepsilon}
\]
can be replaced with 
\[
    L(\varnothing^*).
\]

\begin{theorem}[Equivalence of Regex and Regular Language]
    \[
        \forall L \in \mathbb R,
        \exists \text{ Regex } E \text{ s.t.\ } L(E) = L.
    \]
\end{theorem}

\subsection{GNFA}
% --------------------------------------------------

A generalized nondeterministic finite automaton (GNFA) is an NFA where
\begin{compactitem}
\item there are exactly one arrow entering and one leaving a state,
\item states can be transited using regexes ($\varnothing$ arrows, star arrows, etc.),
\item there are no arrows entering the initial state,
\item there is only one final state, and
\item there are no arrows leaving the final state.
\end{compactitem}

\begin{definition}[GNFA]
    A GNFA is defined as a $5$-tuple
    \[
        \lst{ Q,\Sigma,\delta,s,f }
    \]
    where
    \[
        \delta \colon
        (Q \backslash \set{f}) \times (Q \backslash \set{s})
        \mapsto
        \mathbb R(\Sigma)
    \]
\end{definition}

A GNFA, since it can use transitions with regexes, can help develop a regex of the
language of the GNFA by removing states one at a time until we have the form
\centgraph[3.5cm]{mp/gnfa-0}

The concept of regex finely relates to all automata we've learned so far:
\centgraph[3cm]{mp/relation_automata_regex-0}

