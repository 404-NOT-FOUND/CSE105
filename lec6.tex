
% no Haskell on exams.

the class of regular languages is closed under
\begin{compactitem}
\item union,
%     $ \text{ if } L_1, L_2 \in \mathbb R,
%     \text{ then } L_1 \cup \L_2 \in \mathbb R$;
\item intersection,
%     $ \text{ if } L_1, L_2 \in \mathbb R,
%     \text{ then } L_1 \cap \L_2 \in \mathbb R$;
\item concatenation, and
%     $ \text{ if } L_1, L_2 \in \mathbb R,
%     \text{ then } L_1 \cdot \L_2 \in \mathbb R$;
\item star.
    ($L^* = \set{ w_1w_2 \cdots w_n \mid w_1,w_2,\cdots,w_n \in L, n \ge 0 }$)
%     $ \text{ if } L \in \mathbb R,
%     \text{ then } L^* \in \mathbb R$;
\end{compactitem}

\begin{example}[$\mathbb R$ closed under star]
    \begin{proof}
        use $\varepsilon$ transition from the final states to the initial states
        (including states transited directly from the initial state with an $\varepsilon$
        arrow) to prove that $\mathbb R$ is closed under star.
    \end{proof}
\end{example}

\begin{example}[Build a rather complex language]
    % any number with commas
    \[
        \set{1,\cdots,9} \cdot
        \( \set{0,1,\cdots,9}^* \cdot \set{,} \)^* \cdot
        \set{0,1,\cdots,9}
    \]
\end{example}

\section{Regular Expression}
% ==================================================

\begin{align*}
    E = a                   & L(a) = \set{a}                        \\
    E = \varepsilon         & L(\varepsilon) = \set{\varepsilon}    \\
    E = E_1 \cdot E_2       & L(E) = L(E_1) \cdot L(E_2)            \\
    E = E_1 + E_2           & L(E) = L(E_1) + L(E_2)                \\
    E = (E_1)^*             & L(E) = L\((E_1)^*\) = L(E_1)^*        \\
    E = \varnothing         & L(\varnothing^*)
\end{align*}

In fact, the rule
\[
    E = \varepsilon         & L(\varepsilon) = \set{\varepsilon}    \\
\]
can be replaced with 
\[
    E = \varnothing         & L(\varnothing^*)
\]

\begin{theorem}
    \forall L \in \mathbb R,
    \exists \Regex E \text{ s.t.\ } L(E) = L
\end{theorem}

\begin{definition}[GNFA]
    A GNFA is an automaton where
    \begin{compactitem}
    \item exactly one arrow entering and one leaving a state,
    \item can be transited using $\varnothing$ arrows,
    \item can be transited using star arrows,
    \item no arrows entering the initial state,
    \item only one final state,
    \item no arrow leaving the final state.
    \end{compactitem}

    A GNFA is defined as a $5$-tuple
    \[
        \lst{Q,\Sigma,\delta,s,f}
    \]
    where
    \[
        \delta \colon (Q \backslash \set{f}) \times (Q \backslash \set{s}) 
        \mapsto \mathbb R(\Sigma)
    \]
\end{definition}

A GNFA, since it can use transitions with regular expressions, can help develop a regular
expression of the language of the GNFA by removing each states until we have the form
%% pic s-E->f

% regex -> nfa <-> dfa -> regex
%              -> gnfa -> regex

