
\section{Nondeterministic Finite Automaton (NFA)}
% ==================================================

In automata theory, a finite state machine is called a deterministic finite automaton
(DFA), if
\begin{compactitem}
\item each of its transitions is uniquely determined by its source state and input symbol,
    and
\item reading an input symbol is required for each state transition.
\end{compactitem}

A nondeterministic finite automaton (NFA), or nondeterministic finite state machine,
needn't obey these restrictions. In particular, every DFA is also an NFA.
(via \href{http://en.wikipedia.org/wiki/Nondeterministic\_finite\_automaton}{WikiPedia})

\begin{example}[How many states do you need?]
    Given
    \[
        L = \set{ w \in \set{0,1}^* \mid \text{the $5$th digit of $w$ is $0$} }
    \]
    the DFA is easy to draw:
    \centgraph{mp/nfa-0.pdf}

    What if the digits are counted from the right, i.e.\ the last 5th digit?

    This is where NFAs could be useful. Unlike the earlier language which only takes seven
    states to create a DFA that recognize it, this one could use numbers of states since
    we don't really know how many digits to expect before reaching the one that needs to
    be $0$. An NFA on the other hand, could finish the job with merely six states:
    \centgraph{mp/nfa-1.pdf}
\end{example}

\begin{definition}[NFA]
    An NFA is a $5$-tuple
    \[ \NFA \]
    where
    \begin{compactitem}
    \item $Q$ and $\Sigma$ are finite sets
    \item $s \in Q, F \subseteq Q$
    \item $\delta \colon
        Q \times \Sigma_\varepsilon
        \footnote{$\Sigma_\varepsilon = \Sigma \cup \set\varepsilon$}
        \mapsto \powerset(Q)$
    \end{compactitem}
\end{definition}

\begin{definition}[Configurations of NFA]
    \begin{align*}
        \Conf &= Q \times \Sigma^*
        \\
        I(w)  &= (s, w)
        \\
        H     &= Q \times \set\varepsilon
        \\
        O(q,\varepsilon) &=
                            \begin{cases}
                                \true,  & q \in F  \\
                                \false, & \text{else}
                            \end{cases}
        \\
        R &= \set{
            (q,aw) \mapsto (q',w) \mid
            \forall q \in Q, a \in \Sigma_\varepsilon, w \in \Sigma^*
        }
    \end{align*}
\end{definition}

\begin{definition}[Language of NFA]
    \[
        L(N) = \set{ w \mid \exists \text{ accepting computation on input $w$} }
    \]
\end{definition}

\begin{example}[A simple NFA]

    The following graph shows an automaton $M$ with a binary alphabet that determines if
    the input ends with a $1$.
    \centgraph[4cm]{mp/nfa-2.pdf}

    The automaton $M$ shown above is \emph{not} a DFA since reading a $1$ in state $A$ can
    lead to $A$ or to $B$.

    $M$ in formal notation is
    \[
        M = \set{ \set{A,B}, \set{0,1}, \delta, A, \set{B} }
    \]
    where the transition function $\delta$ can be defined by the state transition table:
    \begin{center} \begin{tabular}{McMcMc}
        \hline
        ~ & 0           & 1             \\
        \hline
        A & \set{A}     & \set{A,B}     \\
        B & \varnothing & \varnothing   \\
        \hline
    \end{tabular} \end{center}
    Note that $\delta(p,1)$ has more than one state---again therefore $M$ is
    nondeterministic.

    Some possible state sequences for the input word ``1011'' are:
    \begin{center} \begin{tabular}{ l *9{Mc} }
        \hline
        input               &   & 1 &   & 0 &   & 1 &   & 1 &   \\
        \hline
        State sequence $1$  & A &   & B &   & ? &   &   &   &   \\
        State sequence $2$  & A &   & A &   & B &   & ? &   &   \\
        State sequence $3$  & A &   & A &   & A &   & B &   & ? \\
        State sequence $4$  & A &   & A &   & A &   & A &   & B \\
        \hline
    \end{tabular} \end{center}

    The word is accepted by $M$ in sequence $4$; it doesn't matter that all other
    sequences fail to do so. In contrast, the word ``10'', which the state sequences
    for are shown below, is rejected by $M$, since there is no way to reach the only
    accepting state, $B$, by reading the final ``0'' symbol or by an
    $\varepsilon$-transition.
    \begin{center} \begin{tabular}{ l *5{Mc} }
        \hline
        input               &   & 1 &   & 0 &     \\
        \hline
        State sequence $1$  & A &   & B &   & ?   \\
        State sequence $2$  & A &   & A &   & A   \\
        \hline
    \end{tabular} \end{center}

    (Example via
    \href{http://en.wikipedia.org/wiki/Nondeterministic\_finite\_automaton}{WikiPedia})
\end{example}

\begin{theorem}
    \label{thm:NFA_to_DFA}
    \begin{align*}
        \forall \text{\upshape DFA} \NFA,  \\
        \exists \text{\upshape DFA} M = \lst{ Q',\Sigma,\delta',s',F' }
        \text{ s.t.\ }
        L(N) = L(M)
    \end{align*}
\end{theorem}

\begin{proof}[Proof of \autoref{thm:NFA_to_DFA}]
    \begin{align*}
        Q' &= \powerset(Q)  \\
        F' &= \set{ A \subseteq Q \mid A \cap F \not= \varnothing }  \\
        s' &= E(\set{s}) = \set{q \in Q \mid \exists s
                                                \xrightarrow\varepsilon q_1
                                                \xrightarrow\varepsilon q_2
                                                \xrightarrow\varepsilon q_3
                                                \cdots
                                                \xrightarrow\varepsilon q
        }  \\
        \delta'(A, a) &= E\( \operatorname*\cup_{q \in A} \delta(q,a) \)
    \end{align*}
\end{proof}

