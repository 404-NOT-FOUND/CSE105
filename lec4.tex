
\section{Nondeterministic Finite Automaton (NFA)}
% ==================================================

\begin{example}[How many states do you need?]
    \[
        L = \set{ w \in \set{0,1}^* \mid \text{the $5$th digit of $w$ is $0$} }
    \]

    \centgraph{mp/dfa-2.pdf}

    What if the digits are counted from the right, i.e.\ the last 5th digit?

    \centgraph{mp/dfa-3.pdf}

\end{example}

\begin{definition}[NFA]
    An NFA is a $5$-tuple
    \[ \NFA \]
    where
    \begin{compactitem}
    \item $Q$ and $\Sigma$ are finite sets
    \item $s \in Q, F \subseteq Q$
    \item $\delta \colon
        Q \times \Sigma_\varepsilon
        \footnote{$\Sigma_\varepsilon = \Sigma \cup \set\varepsilon$}
        \mapsto \powerset(Q)$
    \end{compactitem}

\end{definition}


\begin{definition}[Configurations of NFA]
    \begin{align*}
        \Conf &= Q \times \Sigma^*
        \\
        I(w)  &= (s, w)
        \\
        H     &= Q \times \set\varepsilon
        \\
        O(q,\varepsilon) &=
                            \begin{cases}
                                \true,  & q \in F  \\
                                \false, & \text{else}
                            \end{cases}
        \\
        R &= \set{
            (q,aw) \mapsto (q',w) \mid
            \forall q \in Q, a \in \Sigma_\varepsilon, w \in \Sigma^*
        }
    \end{align*}
\end{definition}

\begin{definition}[Language of NFA]
    \[
        L(N) = \set{ w \mid \exists \text{accepting computation on input $w$} }
    \]
\end{definition}

\begin{theorem}
    \label{thm:NFA_to_DFA}
    \begin{align*}
        \forall \text{\upshape DFA} \NFA,  \\
        \exists \text{\upshape DFA} M = \lst{ Q',\Sigma,\delta',s',F' }
        \text{ s.t.\ }
        L(N) = L(M)
    \end{align*}
\end{theorem}

\begin{proof}[Proof of \autoref{thm:NFA_to_DFA}]
    \begin{align*}
        Q' &= \powerset(Q)  \\
        F' &= \set{ A \subseteq Q \mid A \cap F \not= \varnothing }  \\
        s' &= E(\set{s}) = \set{q \in Q \mid \exists s
                                                \xrightarrow\varepsilon q_1
                                                \xrightarrow\varepsilon q_2
                                                \xrightarrow\varepsilon q_3
                                                \cdots
                                                \xrightarrow\varepsilon q
        }  \\
        \delta'(A, a) &= E( \operatorname*\cup_{q \in A} \delta(q,a) )
    \end{align*}
\end{proof}
