
A subset of $\Sigma^*$ of a DFA that contains all inputs to which the output of the
machine is $\true$ is called the \emph{language} of the machine.

\begin{definition}[Regularity of Language]
    $ L \subseteq \Sigma^* $
    is regular if
    \[
        \exists {\text{\upshape DFA} M} \mid L(M) = L
    \]
\end{definition}

\begin{example}[]
    Given that
    $\varepsilon^* = \set{\varepsilon}$
    and
    $\Sigma^* = \set{\varepsilon,1,0,10,101,\cdots} = \set{0,1}^*$,
    which of the following languages are regular?
    \begin{compactitem}
%     \item
%         $L_1 = \set{ w \in \set{0,1}^* \mid w \text{ is not a multiple of } 3}$,
    \item
        $L_1 = \set{ w \in \set{0,1}^* \mid w \text{ is a power of } 2}$, and
    \item
        $L_2 = \set{ w \in \set{0,1}^* \mid w \text{ is a power of } 3}$.
    \end{compactitem}
    
    $L_1$ is regular while $L_2$ is not. A binary number that is a power of $2$ consists
    of only one $1$ and all other digits should be $0$s. A DFA that recognizes the
    language would be
    \centgraph[7cm]{mp/dfa-1.pdf}

\end{example}

\begin{definition}[Operations on Languages] \ \\
    \begin{compactdesc}
    \item[Complement]
        $L^C = \set{w \in \Sigma^* \mid w \notin L}$
    \item[Union]
        $L_1 \cup L_2 = \set{w \in \Sigma^* \mid w \in L_1 \vee w \in L_2}$
    \item[Intersection]
        $L_1 \cap L_2 = \set{w \mid w \in L_1 \wedge \in L_2}$
    \item[Concatenation]
        $L_1 \cdot L_2 = \set{w_1 \cdot w_2 \mid w \in L_1, w_2 \in L_2}$
    \end{compactdesc}
\end{definition}

\begin{example}[If $L$ is regular, is $L^C$ also regular?]
    Yes.

    \begin{proof}[Proof of statement $\mathbb R$ is closed under complement.]
        Let $L \in \mathbb R,$
        prove $L^C \in \mathbb R:$

        By definition, 
        \[
            \exists \DFA \text{ s.t.\ } L(M) = L.
        \]

        Let $M' = \lst{Q,\Sigma,\delta,s,F^C},$

        then $L(M') = L(M)^C = L^C.$

        $L^C \in \mathbb R$ because $L^C = L(M').$
    \end{proof}
\end{example}

\begin{example}[$\forall L_1, L_2$ \\ 
    $
        L_1 \in \mathbb R \vee L_2 \in \mathbb R
        \implies L_1 \cup L_2 \in \mathbb R
    $]
        Yes, $\mathbb R$ is closed under union.
\end{example}

